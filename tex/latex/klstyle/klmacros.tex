% Partial derivatives
\newcommand{\pd}[2]{\frac{\partial #1}{\partial #2}}
\newcommand{\pdd}[2]{\frac{\partial^2 #1}{\partial #2^2}}
\newcommand{\pddd}[2]{\frac{\partial^3 #1}{\partial #2^3}}
\newcommand{\pmix}[3]{\frac{\partial^2 #1}{\partial #2 \partial #3}}
\newcommand{\pmixx}[4]{\frac{\partial^3 #1}{\partial #2 \partial #3 \partial #4}
}

% Conjugate terms
\newcommand{\CONJ}{\mathrm{c.c.}}

% Differential operators
\newcommand{\polarLapl}[3]{\frac{1}{#2}\pd{}{#2}\left(#2\pd{#1}{#2}\right)%
+\frac{1}{#2^2}\pdd{#1}{#3}}

% Null space
\DeclareMathOperator{\Null}{null}
\DeclareMathOperator{\Range}{range}
\DeclareMathOperator{\Span}{span}
\DeclareMathOperator{\Col}{col}

% Unit vectors
\newcommand{\ehat}{\boldsymbol{\hat e}}
\newcommand{\ihat}{\boldsymbol{\hat \imath}}
\newcommand{\jhat}{\boldsymbol{\hat \jmath}}
\newcommand{\khat}{\boldsymbol{\hat k}}
\newcommand{\xhat}{\boldsymbol{\hat x}}
\newcommand{\yhat}{\boldsymbol{\hat y}}
\newcommand{\zhat}{\boldsymbol{\hat z}}
\newcommand{\rhat}{\boldsymbol{\hat r}}
\newcommand{\Rhat}{\boldsymbol{\hat R}}
\newcommand{\rhohat}{\boldsymbol{\hat\rho}}
\newcommand{\thetahat}{\boldsymbol{\hat\theta}}
\newcommand{\phihat}{\boldsymbol{\hat\phi}}
\newcommand{\erho}{\boldsymbol{\hat e}_\rho}
\newcommand{\ephi}{\boldsymbol{\hat e}_\phi}
\newcommand{\etheta}{\boldsymbol{\hat e}_\theta}
\newcommand{\erhodot}{\boldsymbol{\dot{\hat e}}_\rho}
\newcommand{\ephidot}{\boldsymbol{\dot{\hat e}}_\phi}
\newcommand{\ellhat}{\boldsymbol{\hat \ell}}
\newcommand{\nhat}{\boldsymbol{\hat n}}
\newcommand{\phat}{\boldsymbol{\hat p}}
\newcommand{\qhat}{\boldsymbol{\hat q}}
\newcommand{\tauhat}{\boldsymbol{\hat\tau}}

\newcommand{\bfalpha}{\boldsymbol{\alpha}}
\newcommand{\bfbeta}{\boldsymbol{\beta}}
\newcommand{\bfgamma}{\boldsymbol{\gamma}}
\newcommand{\bfdelta}{\boldsymbol{\delta}}
\newcommand{\bfepsilon}{\boldsymbol{\epsilon}}
\newcommand{\bfzeta}{\boldsymbol{\zeta}}
\newcommand{\bftheta}{\boldsymbol{\theta}}
\newcommand{\bfkappa}{\boldsymbol{\kappa}}
\newcommand{\bflambda}{\boldsymbol{\lambda}}
\newcommand{\bfmu}{\boldsymbol{\mu}}
\newcommand{\bfnu}{\boldsymbol{\nu}}
\newcommand{\bfxi}{\boldsymbol{\xi}}
\newcommand{\bfpi}{\boldsymbol{\pi}}
\newcommand{\bfrho}{\boldsymbol{\rho}}
\newcommand{\bfsigma}{\boldsymbol{\sigma}}
\newcommand{\bftau}{\boldsymbol{\tau}}
\newcommand{\bfupsilon}{\boldsymbol{\upsilon}}
\newcommand{\bfphi}{\boldsymbol{\phi}}
\newcommand{\bfchi}{\boldsymbol{\chi}}
\newcommand{\bfpsi}{\boldsymbol{\psi}}
\newcommand{\bfPsi}{\boldsymbol{\psi}}
\newcommand{\bfomega}{\boldsymbol{\omega}}
\newcommand{\bfOmega}{\boldsymbol{\Omega}}


% Double arrow for transform pairs
\newcommand{\tp}{\longleftrightarrow}

% Scripted Fourier and Laplace transform notation
\newcommand{\FT}[1]{\mathcal{F}\left\{{#1}\right\}}
\newcommand{\LT}[1]{\mathcal{L}\left\{{#1}\right\}}

% Mathcal
\newcommand{\MCL}{\mathcal{L}}

% Functions
\newcommand{\erf}{\mathrm{erf}}


% Ordinary derivatives
\newcommand{\od}[2]{\frac{d #1}{d #2}}
\newcommand{\odd}[2]{\frac{d^2 #1}{d #2^2}}

% Dot notation for time derivatives
\newcommand{\overdot}[1]{\overset{{\mbox{\large\bfseries .}}}{#1}}
\newcommand{\overddot}[1]{\overset{{\mbox{\large\bfseries ..}}}{#1}}

% Common vector space symbols
\newcommand{\R}{\mathbb{R}}
\newcommand{\RR}{\mathbb{R}}
\newcommand{\CC}{\mathbb{C}}
\newcommand{\PP}{\mathbb{P}}
\newcommand{\TT}{\mathbb{T}}
\newcommand{\NN}{\mathbb{N}}
\newcommand{\ZZ}{\mathbb{Z}}
\newcommand{\FF}{\mathbb{F}}
%\newcommand{\null}{\mathrm{null}}

% Script characters
\newcommand{\mcJ}{\mathcal{J}}
\newcommand{\mcK}{\mathcal{K}}
\newcommand{\mcL}{\mathcal{L}}

% Integral over all of R
\newcommand{\intR}{{\int_\infty^\infty}}

% Underbrace an expression
\newcommand{\ub}[2]{\underbrace{{#1}}_{#2}}

% Put a big arrow under an expression
\newcommand{\ua}[1]{\underset{\Longrightarrow}{#1}}

% Put an expression over a long arrow
\newcommand{\oa}[1]{\overset{#1}\longrightarrow}

% Evaluate at limits of integration
\newcommand{\evalat}[1]{\left.#1\right|}

%
%\usepackage{cancel}

% Including multi-page PDF files
\usepackage{pdfpages}
\newcommand\bigpdf[2]{%
    \includepdf[frame=true,pages=1,scale=0.8,pagecommand={#1}]{#2}%
    \includepdf[frame=true,pages=2-,scale=0.85,pagecommand={}]{#2}%
    }

\newcommand\smallpdf[2]{%
    \includepdf[frame=true,pages=1,scale=0.8,pagecommand={#1}]{#2}}



% Circuit drawing
\usepackage{tikz}
\usetikzlibrary{snakes}
\usepackage{circuitikz}
% EMF symbol
\newcommand{\emf}{\mathcal{E}}



% Code listings
\usepackage{listings}
\usepackage{xcolor}
\definecolor{medgray}{RGB}{220,220,220}
\definecolor{lightgray}{RGB}{240,240,240}
\definecolor{bluegray}{RGB}{180,180,200}
\lstset{%
basicstyle={\sffamily\footnotesize},%
language={Mathematica},%
tabsize=2,%
showstringspaces=false,%
frame=shadowbox,%
breaklines=false,%
mathescape=true,%
commentstyle={\rmshape\textcolor{blue}},%
lineskip=1pt,%
aboveskip=\bigskipamount,%
belowskip=\bigskipamount,%
rulesepcolor=\color{bluegray}}

% Highlight an equation
\newcommand{\highlight}[1]{\colorbox{lightgray}{\ensuremath{#1}}}

% Highlighted examples
\usepackage{mdframed}

\ifdefined\example
\let\basicexample\example
\renewenvironment{example}{\begin{mdframed}[backgroundcolor=lightgray,linewidth=2pt]\begin{basicexample}}{\end{basicexample}\end{mdframed}}
\fi

%\ifdefined\chapter
%\ifdefined\example
%\numberwithin{example}{chapter}
%\fi
%\fi

\newcommand\circled[1]{\tikz[baseline=(char.base)]{\node[shape=circle,draw,inner sep=2pt] (char) {#1};}}
